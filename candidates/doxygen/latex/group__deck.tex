\section{\char`\"{}deck\_\-\char`\"{} Deck of cards API}
\label{group__deck}\index{"deck_" Deck of cards API@{""deck\_\-"" Deck of cards API}}
Provides a \char`\"{}deck of cards\char`\"{} random selection interface. 
\subsection*{Functions}
\begin{CompactItemize}
\item 
{\bf T\_\-deck} $\ast$ {\bf deck\_\-create} (size\_\-t size, unsigned int seed, void $\ast$data, size\_\-t dsize)
\begin{CompactList}\small\item\em Allocates an deck of cards and initializes them with ordinal and data.\item\end{CompactList}\item 
void {\bf deck\_\-destroy} ({\bf T\_\-deck} $\ast$d)
\begin{CompactList}\small\item\em Deletes the deck a {\bf all} {\bf cards} {\bf in} {\bf it}. Does not delete data referenced by the cards.\item\end{CompactList}\item 
void {\bf deck\_\-shuffle} ({\bf T\_\-deck} $\ast$d)
\begin{CompactList}\small\item\em shuffles the given {\bf T\_\-deck} {\rm (p.\,\pageref{structT__deck})} d, and resets dealt field. Does not modify the card ordinal or data fields.\item\end{CompactList}\item 
{\bf T\_\-deck\_\-card} $\ast$ {\bf deck\_\-deal} ({\bf T\_\-deck} $\ast$d, int blackjack)
\begin{CompactList}\small\item\em returns the next card in the deck, advances the dealt field, automatically reshuffling, as needed unless \char`\"{}blackjack\char`\"{} mode is set\item\end{CompactList}\item 
{\bf T\_\-deck\_\-card} $\ast$ {\bf deck\_\-peek} ({\bf T\_\-deck} $\ast$d, int blackjack)
\begin{CompactList}\small\item\em returns the next card in the deck without advancing \char`\"{}dealt\char`\"{}. Allows the caller peek at the next card without actually dealing it. This can result in the deck shuffling unless blackjack is set, or NULL if blackjack is set and the deck is out of cards.\item\end{CompactList}\end{CompactItemize}


\subsection{Detailed Description}
Provides a \char`\"{}deck of cards\char`\"{} random selection interface.

\char`\"{}Deck of cards\char`\"{} randomization ensures that all elements of a population are used once before any are used more than once. This is useful when an exact population profile is desired, not just one the is probablistic over a sufficient sample set.

A {\bf T\_\-deck} {\rm (p.\,\pageref{structT__deck})} contains the deck state and an array of {\bf T\_\-deck\_\-card} {\rm (p.\,\pageref{structT__deck__card})} cards. By default the {\bf T\_\-deck\_\-card} {\rm (p.\,\pageref{structT__deck__card})} structs contain a unique ordinal and a data pointer into a data array of constant stride (or NULL if not supplied).

Controlling the population

The {\bf T\_\-deck} {\rm (p.\,\pageref{structT__deck})} can be stacked by altering either the distribution of the ordinals or of the data pointers. The size of the {\bf T\_\-deck} {\rm (p.\,\pageref{structT__deck})} should be defined as the size of the total population, and the appropriate number of cards set to match the population for each value of the distribution

 TODO\begin{CompactItemize}
\item 
Another creator which would stack the deck according to a supplied distribution would be handy. (simple examples would include a multi-deck \char`\"{}shoe\char`\"{})\item 
Use of a better random number generator (rand\_\-r is weak).\item 
Repeatable shuffles (return back the seed state that started the deck)\item 
Enhancement to \char`\"{}blackjack\char`\"{} mode which would reshuffle when the deck reaches some \char`\"{}almost dealt\char`\"{} state (which violates the logic of have a deck, unless you {\bf really} do want to use it for a game of chance ;-) ) \end{CompactItemize}


\subsection{Function Documentation}
\index{deck@{deck}!deck_create@{deck\_\-create}}
\index{deck_create@{deck\_\-create}!deck@{deck}}
\subsubsection{\setlength{\rightskip}{0pt plus 5cm}{\bf T\_\-deck}$\ast$ deck\_\-create (size\_\-t {\em size}, unsigned int {\em seed}, void $\ast$ {\em data}, size\_\-t {\em dsize})}\label{group__deck_a0}


Allocates an deck of cards and initializes them with ordinal and data.

The ord (ordinal) field of each card is set to a zero based index of the  card in the deck of card array. This value can freely be changed to suit the needs of the calling program, and is initialized as a convenience.

If parameter {\bf data} is non-null, the data field of each card is set to data + ordinal $\ast$ dsize. Thus, each card points to consectutive entries in an array of data elements of size dsize. Again this data can be freely modified after create to suit the needs of the calling program\begin{Desc}
\item[Parameters: ]\par
\begin{description}
\item[{\em 
size}]number of cards in the deck created \item[{\em 
seed}]the seed to use for this deck \item[{\em 
data}]array of data to map to {\bf T\_\-deck\_\-card::data} {\rm (p.\,\pageref{structT__deck__card_m1})}, NULL indicates none \item[{\em 
dsize}]the stride of the data array (if non-NULL) \end{description}
\end{Desc}
\begin{Desc}
\item[Returns: ]\par
the created deck \end{Desc}
\index{deck@{deck}!deck_deal@{deck\_\-deal}}
\index{deck_deal@{deck\_\-deal}!deck@{deck}}
\subsubsection{\setlength{\rightskip}{0pt plus 5cm}{\bf T\_\-deck\_\-card}$\ast$ deck\_\-deal ({\bf T\_\-deck} $\ast$ {\em d}, int {\em blackjack})}\label{group__deck_a3}


returns the next card in the deck, advances the dealt field, automatically reshuffling, as needed unless \char`\"{}blackjack\char`\"{} mode is set

\begin{Desc}
\item[Parameters: ]\par
\begin{description}
\item[{\em 
d}]the deck to deal \item[{\em 
blackjack}]don't shuffle at end of deck, instead return NULL \end{description}
\end{Desc}
\begin{Desc}
\item[Returns: ]\par
the card on the top of the deck or NULL if blackjack and out of cards. \end{Desc}
\index{deck@{deck}!deck_destroy@{deck\_\-destroy}}
\index{deck_destroy@{deck\_\-destroy}!deck@{deck}}
\subsubsection{\setlength{\rightskip}{0pt plus 5cm}void deck\_\-destroy ({\bf T\_\-deck} $\ast$ {\em d})}\label{group__deck_a1}


Deletes the deck a {\bf all} {\bf cards} {\bf in} {\bf it}. Does not delete data referenced by the cards.

\begin{Desc}
\item[Parameters: ]\par
\begin{description}
\item[{\em 
d}]the deck delete \end{description}
\end{Desc}
\index{deck@{deck}!deck_peek@{deck\_\-peek}}
\index{deck_peek@{deck\_\-peek}!deck@{deck}}
\subsubsection{\setlength{\rightskip}{0pt plus 5cm}{\bf T\_\-deck\_\-card}$\ast$ deck\_\-peek ({\bf T\_\-deck} $\ast$ {\em d}, int {\em blackjack})}\label{group__deck_a4}


returns the next card in the deck without advancing \char`\"{}dealt\char`\"{}. Allows the caller peek at the next card without actually dealing it. This can result in the deck shuffling unless blackjack is set, or NULL if blackjack is set and the deck is out of cards.

\begin{Desc}
\item[Parameters: ]\par
\begin{description}
\item[{\em 
d}]the deck to deal \item[{\em 
blackjack}]don't shuffle at end of deck, instead return NULL \end{description}
\end{Desc}
\begin{Desc}
\item[Returns: ]\par
the card on the top of the deck \end{Desc}
\index{deck@{deck}!deck_shuffle@{deck\_\-shuffle}}
\index{deck_shuffle@{deck\_\-shuffle}!deck@{deck}}
\subsubsection{\setlength{\rightskip}{0pt plus 5cm}void deck\_\-shuffle ({\bf T\_\-deck} $\ast$ {\em d})}\label{group__deck_a2}


shuffles the given {\bf T\_\-deck} {\rm (p.\,\pageref{structT__deck})} d, and resets dealt field. Does not modify the card ordinal or data fields.

\begin{Desc}
\item[Parameters: ]\par
\begin{description}
\item[{\em 
d}]the deck to shuffle \end{description}
\end{Desc}
