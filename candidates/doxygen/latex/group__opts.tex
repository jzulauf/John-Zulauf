\section{\char`\"{}opts\_\-\char`\"{} Callback based command line options handler}
\label{group__opts}\index{"opts_" Callback based command line options handler@{""opts\_\-"" Callback based command line options handler}}
This API wraps the getopt\_\-long functionality and provides a general call back mechanism to define the behavior of command line flags. 
\subsection*{Modules}
\begin{CompactItemize}
\item 
{\bf Static variables and functions (internal)}
\item 
{\bf Standard callback actions}
\end{CompactItemize}
\subsection*{Functions}
\begin{CompactItemize}
\item 
void {\bf opts\_\-init} (const char $\ast$progname, const char $\ast$progversion)
\begin{CompactList}\small\item\em initialize the {\bf \char`\"{}opts\_\-\char`\"{} Callback based command line options handler} {\rm (p.\,\pageref{group__opts})} subsystem. Call this before calling\item\end{CompactList}\item 
void {\bf opts\_\-def} (char short\_\-name, char $\ast$long\_\-name, char $\ast$description, char $\ast$arg\_\-name, {\bf T\_\-opts\_\-action} action, void $\ast$data)
\begin{CompactList}\small\item\em add the given option to the opts\_\-table\item\end{CompactList}\item 
\index{opts_default_opts@{opts\_\-default\_\-opts}!opts@{opts}}\index{opts@{opts}!opts_default_opts@{opts\_\-default\_\-opts}}
void {\bf opts\_\-default\_\-opts} ()\label{group__opts_a2}

\begin{CompactList}\small\item\em add the default opts to opts\_\-table\item\end{CompactList}\item 
int {\bf opts\_\-getopt} (int argc, char $\ast$const $\ast$argv)
\begin{CompactList}\small\item\em parse the known options, calling their callback actions, and returning on unknown options with the identical semantics of getopt\item\end{CompactList}\item 
\index{opts_usage@{opts\_\-usage}!opts@{opts}}\index{opts@{opts}!opts_usage@{opts\_\-usage}}
void {\bf opts\_\-usage} (void)\label{group__opts_a4}

\begin{CompactList}\small\item\em output the usage message and exit\item\end{CompactList}\end{CompactItemize}


\subsection{Detailed Description}
This API wraps the getopt\_\-long functionality and provides a general call back mechanism to define the behavior of command line flags.

{\bf TO} {\bf USE}:\begin{CompactItemize}
\item 
{\bf opts\_\-init()} {\rm (p.\,\pageref{group__opts_a0})} to initialize the subsystem \item 
{\bf opts\_\-default\_\-opts()} {\rm (p.\,\pageref{group__opts_a2})} to add useful default opts (help and version)\item 
Define the arguments using:\begin{CompactItemize}
\item 
{\bf opts\_\-def()} {\rm (p.\,\pageref{group__opts_a1})}\item 
{\bf OPTS\_\-DEF} {\rm (p.\,\pageref{opts__utils_8h_a3})} \item 
{\bf OPTS\_\-DEF\_\-ARG} {\rm (p.\,\pageref{opts__utils_8h_a4})}, \item 
{\bf OPTS\_\-DEF\_\-CLEAR} {\rm (p.\,\pageref{opts__utils_8h_a6})} \item 
{\bf OPTS\_\-DEF\_\-INCR} {\rm (p.\,\pageref{opts__utils_8h_a7})} \item 
{\bf OPTS\_\-DEF\_\-SET} {\rm (p.\,\pageref{opts__utils_8h_a8})} \item 
{\bf OPTS\_\-DEF\_\-STRTOL} {\rm (p.\,\pageref{opts__utils_8h_a5})} \end{CompactItemize}
\item 
{\bf opts\_\-getopt()} {\rm (p.\,\pageref{group__opts_a3})} parses the known options\begin{CompactItemize}
\item 
-1 end of options (no error)\item 
Anything else, and unknown flag either\begin{CompactItemize}
\item 
catch in a traditional getopts switch such as 

\footnotesize\begin{verbatim}       while ((c = opts_getopt (argc, argv)) != -1) {
               switch (c) {
               // add stuff opts_getopt() can't handle
               case 'Z':
                       Z_option_handler();
                       break;
               // end of stuff opts_utils can't handle 
               default:
                       opts_usage();
               }
       }
\end{verbatim}\normalsize 
\item 
{\bf opts\_\-usage()} {\rm (p.\,\pageref{group__opts_a4})} to display usage message and exit.\end{CompactItemize}
\end{CompactItemize}
\end{CompactItemize}
{\bf TODO}\begin{CompactItemize}
\item 
Add optional argument support\item 
Allow use of non-printable shortnames for longname only options \end{CompactItemize}


\subsection{Function Documentation}
\index{opts@{opts}!opts_def@{opts\_\-def}}
\index{opts_def@{opts\_\-def}!opts@{opts}}
\subsubsection{\setlength{\rightskip}{0pt plus 5cm}void opts\_\-def (char {\em short\_\-name}, char $\ast$ {\em long\_\-name}, char $\ast$ {\em description}, char $\ast$ {\em arg\_\-name}, {\bf T\_\-opts\_\-action} {\em action}, void $\ast$ {\em data})}\label{group__opts_a1}


add the given option to the opts\_\-table

\begin{Desc}
\item[Parameters: ]\par
\begin{description}
\item[{\em 
short\_\-name}]the 1 character that uniquely identifies this option \item[{\em 
long\_\-name}]the long (no spaces) name for this option. Must be {\bf non-NULL}. \item[{\em 
description}]descriptive text for this option. If arg\_\-name is non-NULL, \%s in description will insert arg\_\-name. \item[{\em 
arg\_\-name}]the descriptive name of the argument the option takes (if an option takes an argument), or NULL if it doesn't. \item[{\em 
action}]the callback action if the flag is found \item[{\em 
data}]pointer to the data the callback will need. \end{description}
\end{Desc}
\index{opts@{opts}!opts_getopt@{opts\_\-getopt}}
\index{opts_getopt@{opts\_\-getopt}!opts@{opts}}
\subsubsection{\setlength{\rightskip}{0pt plus 5cm}int opts\_\-getopt (int {\em argc}, char $\ast$const $\ast$ {\em argv})}\label{group__opts_a3}


parse the known options, calling their callback actions, and returning on unknown options with the identical semantics of getopt

\begin{Desc}
\item[Parameters: ]\par
\begin{description}
\item[{\em 
argc}]the argument count \item[{\em 
argv}]the argument array \end{description}
\end{Desc}
\begin{Desc}
\item[Returns: ]\par
-1 for end of arguments, anything else for an unknown options \end{Desc}
\index{opts@{opts}!opts_init@{opts\_\-init}}
\index{opts_init@{opts\_\-init}!opts@{opts}}
\subsubsection{\setlength{\rightskip}{0pt plus 5cm}void opts\_\-init (const char $\ast$ {\em progname}, const char $\ast$ {\em progversion})}\label{group__opts_a0}


initialize the {\bf \char`\"{}opts\_\-\char`\"{} Callback based command line options handler} {\rm (p.\,\pageref{group__opts})} subsystem. Call this before calling

\begin{Desc}
\item[Returns: ]\par
0 for success non-zero for failure \end{Desc}
