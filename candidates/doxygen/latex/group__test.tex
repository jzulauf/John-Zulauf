\section{\char`\"{}test\_\-\char`\"{} Test framework for performance testing}
\label{group__test}\index{"test_" Test framework for performance testing@{""test\_\-"" Test framework for performance testing}}
\subsection*{Modules}
\begin{CompactItemize}
\item 
{\bf Test application state management}
\item 
{\bf Utility routines (cache, stride, alignment, and fill)}
\item 
{\bf Test buffer creation and management functions}
\end{CompactItemize}
\subsection*{Functions}
\begin{CompactItemize}
\item 
{\bf T\_\-test\_\-app} $\ast$ {\bf test\_\-init} (const char $\ast$progname, const char $\ast$progversion, {\bf T\_\-test\_\-type} type)
\begin{CompactList}\small\item\em Initializes the app state structure and command line (opts table).\item\end{CompactList}\item 
void {\bf test\_\-create\_\-buffers} ({\bf T\_\-test\_\-app} $\ast$app, size\_\-t scale, size\_\-t extra)
\begin{CompactList}\small\item\em Allocates the default set of buffers {\bf NOTE}: init buffers computes and allocates the buffers and the TOC but {\bf doesn't} fill them. It {\bf does} look at the {\bf T\_\-test\_\-app} {\rm (p.\,\pageref{structT__test__app})} settings and the test\_\-type to initialize the count and wrap call backs.\item\end{CompactList}\item 
int {\bf test\_\-action\_\-oneshot} (void $\ast$ptr\_\-to\_\-test\_\-app)
\begin{CompactList}\small\item\em Callback for {\bf opts\_\-getopt()} {\rm (p.\,\pageref{group__opts_a3})} and the --oneshot flag.\item\end{CompactList}\end{CompactItemize}


\subsection{Function Documentation}
\index{test@{test}!test_action_oneshot@{test\_\-action\_\-oneshot}}
\index{test_action_oneshot@{test\_\-action\_\-oneshot}!test@{test}}
\subsubsection{\setlength{\rightskip}{0pt plus 5cm}int test\_\-action\_\-oneshot (void $\ast$ {\em ptr\_\-to\_\-test\_\-app})}\label{group__test_a2}


Callback for {\bf opts\_\-getopt()} {\rm (p.\,\pageref{group__opts_a3})} and the --oneshot flag.

\begin{Desc}
\item[Parameters: ]\par
\begin{description}
\item[{\em 
ptr\_\-to\_\-test\_\-app}]action callback data pointing to the {\bf T\_\-test\_\-app} {\rm (p.\,\pageref{structT__test__app})} structure \end{description}
\end{Desc}
\begin{Desc}
\item[Returns: ]\par
0 (action successful) \end{Desc}
\index{test@{test}!test_create_buffers@{test\_\-create\_\-buffers}}
\index{test_create_buffers@{test\_\-create\_\-buffers}!test@{test}}
\subsubsection{\setlength{\rightskip}{0pt plus 5cm}void test\_\-create\_\-buffers ({\bf T\_\-test\_\-app} $\ast$ {\em app}, size\_\-t {\em scale}, size\_\-t {\em extra})}\label{group__test_a1}


Allocates the default set of buffers {\bf NOTE}: init buffers computes and allocates the buffers and the TOC but {\bf doesn't} fill them. It {\bf does} look at the {\bf T\_\-test\_\-app} {\rm (p.\,\pageref{structT__test__app})} settings and the test\_\-type to initialize the count and wrap call backs.

\begin{Desc}
\item[Parameters: ]\par
\begin{description}
\item[{\em 
app}]the application state \item[{\em 
scale}]size multiplier the samples: samp\_\-size = count $\ast$ scale + extra \item[{\em 
extra}]additional bytes for each sample (see scale parameter) \end{description}
\end{Desc}
\index{test@{test}!test_init@{test\_\-init}}
\index{test_init@{test\_\-init}!test@{test}}
\subsubsection{\setlength{\rightskip}{0pt plus 5cm}{\bf T\_\-test\_\-app}$\ast$ test\_\-init (const char $\ast$ {\em progname}, const char $\ast$ {\em progversion}, {\bf T\_\-test\_\-type} {\em type})}\label{group__test_a0}


Initializes the app state structure and command line (opts table).

\begin{Desc}
\item[Parameters: ]\par
\begin{description}
\item[{\em 
progname}]name of this program (information only) \item[{\em 
progversion}]name of this program (for help and usage) \item[{\em 
type}]the cannonical form of the test \end{description}
\end{Desc}
