\section{/users/jcrouse/perforce/drivers/projects/geodelx/linux/glibc/candidates/common/include/table.h File Reference}
\label{table_8h}\index{/users/jcrouse/perforce/drivers/projects/geodelx/linux/glibc/candidates/common/include/table.h@{/users/jcrouse/perforce/drivers/projects/geodelx/linux/glibc/candidates/common/include/table.h}}
Types and default settings for {\bf \char`\"{}table\_\-\char`\"{} Simple table utility API} {\rm (p.\,\pageref{group__simple__table})} and {\bf \char`\"{}cell\_\-\char`\"{} A smart union datatype} {\rm (p.\,\pageref{group__table__cell})}. 


{\tt \#include $<$stdio.h$>$}\par
{\tt \#include \char`\"{}zstring.h\char`\"{}}\par
\subsection*{Compounds}
\begin{CompactItemize}
\item 
struct {\bf S\_\-cell}
\item 
struct {\bf S\_\-cell\_\-class}
\item 
struct {\bf S\_\-cell\_\-fmt}
\item 
struct {\bf S\_\-table}
\item 
struct {\bf S\_\-table\_\-class}
\item 
struct {\bf S\_\-table\_\-fmt}
\item 
union {\bf U\_\-cell\_\-data}
\end{CompactItemize}
\subsection*{Typedefs}
\begin{CompactItemize}
\item 
typedef enum {\bf E\_\-cell\_\-type} {\bf T\_\-cell\_\-type}
\item 
typedef {\bf S\_\-cell\_\-fmt} {\bf T\_\-cell\_\-fmt}
\item 
typedef {\bf S\_\-cell\_\-class} {\bf T\_\-cell\_\-class}
\item 
typedef {\bf U\_\-cell\_\-data} {\bf T\_\-cell\_\-data}
\item 
typedef {\bf S\_\-cell} {\bf T\_\-cell}
\item 
typedef {\bf S\_\-table\_\-fmt} {\bf T\_\-table\_\-fmt}
\item 
typedef {\bf S\_\-table\_\-class} {\bf T\_\-table\_\-class}
\item 
typedef {\bf S\_\-table} {\bf T\_\-table}
\end{CompactItemize}
\subsection*{Enumerations}
\begin{CompactItemize}
\item 
enum {\bf E\_\-cell\_\-type} \{ {\bf CELL\_\-UNINIT} =  0, 
{\bf CELL\_\-DOUBLE}, 
{\bf CELL\_\-LONG}, 
{\bf CELL\_\-ULONG}, 
{\bf CELL\_\-LONGLONG}, 
{\bf CELL\_\-ULONGLONG}, 
{\bf CELL\_\-VOID}, 
{\bf CELL\_\-ZSTRING}
 \}
\end{CompactItemize}
\subsection*{Functions}
\begin{CompactItemize}
\item 
{\bf T\_\-cell} $\ast$ {\bf cell\_\-create\_\-array} (size\_\-t num)
\item 
void {\bf cell\_\-destroy\_\-array} ({\bf T\_\-cell} $\ast$cell\_\-array, size\_\-t num)
\item 
{\bf T\_\-cell} $\ast$ {\bf cell\_\-set} ({\bf T\_\-cell} $\ast$c, {\bf T\_\-cell\_\-type} type, const void $\ast$data)
\item 
int {\bf cell\_\-fprintf} ({\bf T\_\-cell} $\ast$c, FILE $\ast$out, {\bf T\_\-cell\_\-fmt} $\ast$fmt)
\item 
int {\bf cell\_\-sprintf} ({\bf T\_\-cell} $\ast$c, {\bf T\_\-zstring} $\ast$z, {\bf T\_\-cell\_\-fmt} $\ast$fmt)
\item 
{\bf T\_\-table} $\ast$ {\bf table\_\-create} (size\_\-t rows, size\_\-t cols)
\item 
void {\bf table\_\-destroy} ({\bf T\_\-table} $\ast$t)
\item 
{\bf T\_\-cell} $\ast$ {\bf table\_\-get\_\-cell} ({\bf T\_\-table} $\ast$t, size\_\-t row, size\_\-t col)
\item 
{\bf T\_\-cell} $\ast$ {\bf table\_\-set\_\-cell} ({\bf T\_\-table} $\ast$t, size\_\-t row, size\_\-t col, {\bf T\_\-cell\_\-type} type, const void $\ast$data)
\item 
{\bf T\_\-cell} $\ast$ {\bf table\_\-set\_\-cell\_\-zstr} ({\bf T\_\-table} $\ast$t, size\_\-t row, size\_\-t col, const char $\ast$zstr)
\item 
{\bf T\_\-cell} $\ast$ {\bf table\_\-set\_\-cell\_\-dbl} ({\bf T\_\-table} $\ast$t, size\_\-t row, size\_\-t col, double dbl)
\item 
{\bf T\_\-cell} $\ast$ {\bf table\_\-set\_\-cell\_\-u64} ({\bf T\_\-table} $\ast$t, size\_\-t row, size\_\-t col, unsigned long long u64)
\item 
{\bf T\_\-cell} $\ast$ {\bf table\_\-set\_\-cell\_\-u32} ({\bf T\_\-table} $\ast$t, size\_\-t row, size\_\-t col, unsigned long u32)
\item 
{\bf T\_\-cell} $\ast$ {\bf table\_\-reformat} ({\bf T\_\-table} $\ast$t, size\_\-t row, size\_\-t col, {\bf T\_\-cell\_\-fmt} $\ast$fmt)
\item 
void {\bf table\_\-print} ({\bf T\_\-table} $\ast$t, FILE $\ast$out, {\bf T\_\-table\_\-fmt} $\ast$fmt)
\end{CompactItemize}
\subsection*{Variables}
\begin{CompactItemize}
\item 
{\bf T\_\-cell\_\-fmt} {\bf g\_\-cell\_\-fmt\_\-default}
\item 
{\bf T\_\-cell\_\-class} {\bf g\_\-cell\_\-class}
\item 
{\bf T\_\-table\_\-fmt} {\bf g\_\-table\_\-fmt\_\-html}
\item 
{\bf T\_\-table\_\-fmt} {\bf g\_\-table\_\-fmt\_\-csv}
\item 
{\bf T\_\-table\_\-class} {\bf g\_\-table\_\-class}
\end{CompactItemize}


\subsection{Detailed Description}
Types and default settings for {\bf \char`\"{}table\_\-\char`\"{} Simple table utility API} {\rm (p.\,\pageref{group__simple__table})} and {\bf \char`\"{}cell\_\-\char`\"{} A smart union datatype} {\rm (p.\,\pageref{group__table__cell})}.





\subsection{Typedef Documentation}
\index{table.h@{table.h}!T_cell@{T\_\-cell}}
\index{T_cell@{T\_\-cell}!table.h@{table.h}}
\subsubsection{\setlength{\rightskip}{0pt plus 5cm}typedef struct {\bf S\_\-cell}  T\_\-cell}\label{table_8h_a4}


The data defining a T\_\-cell:: instance

The use of capitilized names makes it convenient to access in macros using CELL\_\-\#field to indicate the type for the field \index{table.h@{table.h}!T_cell_class@{T\_\-cell\_\-class}}
\index{T_cell_class@{T\_\-cell\_\-class}!table.h@{table.h}}
\subsubsection{\setlength{\rightskip}{0pt plus 5cm}typedef struct {\bf S\_\-cell\_\-class}  T\_\-cell\_\-class}\label{table_8h_a2}


The class data for T\_\-cell:: \index{table.h@{table.h}!T_cell_data@{T\_\-cell\_\-data}}
\index{T_cell_data@{T\_\-cell\_\-data}!table.h@{table.h}}
\subsubsection{\setlength{\rightskip}{0pt plus 5cm}typedef union {\bf U\_\-cell\_\-data}  T\_\-cell\_\-data}\label{table_8h_a3}


The flexible storage union for T\_\-cell:: \index{table.h@{table.h}!T_cell_fmt@{T\_\-cell\_\-fmt}}
\index{T_cell_fmt@{T\_\-cell\_\-fmt}!table.h@{table.h}}
\subsubsection{\setlength{\rightskip}{0pt plus 5cm}typedef struct {\bf S\_\-cell\_\-fmt}  T\_\-cell\_\-fmt}\label{table_8h_a1}


The formats to use when printing cell data, one for each type in T\_\-cell\_\-type:: \index{table.h@{table.h}!T_cell_type@{T\_\-cell\_\-type}}
\index{T_cell_type@{T\_\-cell\_\-type}!table.h@{table.h}}
\subsubsection{\setlength{\rightskip}{0pt plus 5cm}typedef enum {\bf E\_\-cell\_\-type}  T\_\-cell\_\-type}\label{table_8h_a0}


The list of valid types for {\bf T\_\-cell::data} {\rm (p.\,\pageref{structS__cell_m0})}, stored in {\bf T\_\-cell::type} {\rm (p.\,\pageref{structS__cell_m1})} \index{table.h@{table.h}!T_table@{T\_\-table}}
\index{T_table@{T\_\-table}!table.h@{table.h}}
\subsubsection{\setlength{\rightskip}{0pt plus 5cm}typedef struct {\bf S\_\-table}  T\_\-table}\label{table_8h_a9}


the data defining a T\_\-table:: instance \index{table.h@{table.h}!T_table_class@{T\_\-table\_\-class}}
\index{T_table_class@{T\_\-table\_\-class}!table.h@{table.h}}
\subsubsection{\setlength{\rightskip}{0pt plus 5cm}typedef struct {\bf S\_\-table\_\-class}  T\_\-table\_\-class}\label{table_8h_a8}


class data for T\_\-table:: \index{table.h@{table.h}!T_table_fmt@{T\_\-table\_\-fmt}}
\index{T_table_fmt@{T\_\-table\_\-fmt}!table.h@{table.h}}
\subsubsection{\setlength{\rightskip}{0pt plus 5cm}typedef struct {\bf S\_\-table\_\-fmt}  T\_\-table\_\-fmt}\label{table_8h_a7}


Store the strings to control formatting of T\_\-table:: 

\subsection{Enumeration Type Documentation}
\index{table.h@{table.h}!E_cell_type@{E\_\-cell\_\-type}}
\index{E_cell_type@{E\_\-cell\_\-type}!table.h@{table.h}}
\subsubsection{\setlength{\rightskip}{0pt plus 5cm}enum E\_\-cell\_\-type}\label{table_8h_a37}


The list of valid types for {\bf T\_\-cell::data} {\rm (p.\,\pageref{structS__cell_m0})}, stored in {\bf T\_\-cell::type} {\rm (p.\,\pageref{structS__cell_m1})} \begin{Desc}
\item[Enumeration values: ]\par
\begin{description}
\index{CELL_UNINIT@{CELL\_\-UNINIT}!table.h@{table.h}}\index{table.h@{table.h}!CELL_UNINIT@{CELL\_\-UNINIT}}\item[{\em 
{\em CELL\_\-UNINIT}\label{table_8h_a37a14}
}]{\bf T\_\-cell::data} {\rm (p.\,\pageref{structS__cell_m0})} is uninitialized \index{CELL_DOUBLE@{CELL\_\-DOUBLE}!table.h@{table.h}}\index{table.h@{table.h}!CELL_DOUBLE@{CELL\_\-DOUBLE}}\item[{\em 
{\em CELL\_\-DOUBLE}\label{table_8h_a37a15}
}]{\bf T\_\-cell::data} {\rm (p.\,\pageref{structS__cell_m0})} is a double \index{CELL_LONG@{CELL\_\-LONG}!table.h@{table.h}}\index{table.h@{table.h}!CELL_LONG@{CELL\_\-LONG}}\item[{\em 
{\em CELL\_\-LONG}\label{table_8h_a37a16}
}]{\bf T\_\-cell::data} {\rm (p.\,\pageref{structS__cell_m0})} is a long \index{CELL_ULONG@{CELL\_\-ULONG}!table.h@{table.h}}\index{table.h@{table.h}!CELL_ULONG@{CELL\_\-ULONG}}\item[{\em 
{\em CELL\_\-ULONG}\label{table_8h_a37a17}
}]{\bf T\_\-cell::data} {\rm (p.\,\pageref{structS__cell_m0})} is an unsigned long \index{CELL_LONGLONG@{CELL\_\-LONGLONG}!table.h@{table.h}}\index{table.h@{table.h}!CELL_LONGLONG@{CELL\_\-LONGLONG}}\item[{\em 
{\em CELL\_\-LONGLONG}\label{table_8h_a37a18}
}]{\bf T\_\-cell::data} {\rm (p.\,\pageref{structS__cell_m0})} is a long long \index{CELL_ULONGLONG@{CELL\_\-ULONGLONG}!table.h@{table.h}}\index{table.h@{table.h}!CELL_ULONGLONG@{CELL\_\-ULONGLONG}}\item[{\em 
{\em CELL\_\-ULONGLONG}\label{table_8h_a37a19}
}]{\bf T\_\-cell::data} {\rm (p.\,\pageref{structS__cell_m0})} is an unsigned long long \index{CELL_VOID@{CELL\_\-VOID}!table.h@{table.h}}\index{table.h@{table.h}!CELL_VOID@{CELL\_\-VOID}}\item[{\em 
{\em CELL\_\-VOID}\label{table_8h_a37a20}
}]{\bf T\_\-cell::data} {\rm (p.\,\pageref{structS__cell_m0})} is a void $\ast$ \index{CELL_ZSTRING@{CELL\_\-ZSTRING}!table.h@{table.h}}\index{table.h@{table.h}!CELL_ZSTRING@{CELL\_\-ZSTRING}}\item[{\em 
{\em CELL\_\-ZSTRING}\label{table_8h_a37a21}
}]{\bf T\_\-cell::data} {\rm (p.\,\pageref{structS__cell_m0})} is a string (T\_\-zstring:: ) \end{description}
\end{Desc}



\subsection{Function Documentation}
\index{table.h@{table.h}!cell_create_array@{cell\_\-create\_\-array}}
\index{cell_create_array@{cell\_\-create\_\-array}!table.h@{table.h}}
\subsubsection{\setlength{\rightskip}{0pt plus 5cm}{\bf T\_\-cell}$\ast$ cell\_\-create\_\-array (size\_\-t {\em num})}\label{table_8h_a22}


Array creator for T\_\-cell::

Since the \char`\"{}uninitialized\char`\"{} type is 0, we simply calloc to create\begin{Desc}
\item[Parameters: ]\par
\begin{description}
\item[{\em 
num}]the number of T\_\-cell:: to create \end{description}
\end{Desc}
\begin{Desc}
\item[Returns: ]\par
the array of T\_\-cell:: created \end{Desc}
\index{table.h@{table.h}!cell_destroy_array@{cell\_\-destroy\_\-array}}
\index{cell_destroy_array@{cell\_\-destroy\_\-array}!table.h@{table.h}}
\subsubsection{\setlength{\rightskip}{0pt plus 5cm}void cell\_\-destroy\_\-array ({\bf T\_\-cell} $\ast$ {\em cell\_\-array}, size\_\-t {\em num})}\label{table_8h_a23}


Destructor for arrays of cells

WARNING: num must match the number passed to the creator \index{table.h@{table.h}!cell_fprintf@{cell\_\-fprintf}}
\index{cell_fprintf@{cell\_\-fprintf}!table.h@{table.h}}
\subsubsection{\setlength{\rightskip}{0pt plus 5cm}int cell\_\-fprintf ({\bf T\_\-cell} $\ast$ {\em c}, FILE $\ast$ {\em out}, {\bf T\_\-cell\_\-fmt} $\ast$ {\em fmt})}\label{table_8h_a25}


fprintf the cell to the given FILE\begin{Desc}
\item[Parameters: ]\par
\begin{description}
\item[{\em 
c}]the cell to fprintf \item[{\em 
out}]the file to write \item[{\em 
fmt}]the cell format to use, or NULL for to use {\bf T\_\-cell\_\-class::fmt} {\rm (p.\,\pageref{structS__cell__class_m0})} \end{description}
\end{Desc}
\begin{Desc}
\item[Returns: ]\par
the return value fprintf \end{Desc}
\index{table.h@{table.h}!cell_set@{cell\_\-set}}
\index{cell_set@{cell\_\-set}!table.h@{table.h}}
\subsubsection{\setlength{\rightskip}{0pt plus 5cm}{\bf T\_\-cell}$\ast$ cell\_\-set ({\bf T\_\-cell} $\ast$ {\em c}, {\bf T\_\-cell\_\-type} {\em type}, const void $\ast$ {\em data})}\label{table_8h_a24}


Sets the value\begin{Desc}
\item[Parameters: ]\par
\begin{description}
\item[{\em 
c}]the cell to set (if NULL, function is NOOP) \item[{\em 
type}]the type of data \item[{\em 
data}]the address of the data to store (except for type VOID where the parameter is the data to store). ZSTRING are copied into a T\_\-zstring::. \end{description}
\end{Desc}
\begin{Desc}
\item[Returns: ]\par
the cell itself (can be NULL)\end{Desc}
Note: changing a cell from type ZSTRING destroys the T\_\-zstring:: referenced do {\bf not} keep a pointer to the T\_\-zstring:: in your own code. \index{table.h@{table.h}!cell_sprintf@{cell\_\-sprintf}}
\index{cell_sprintf@{cell\_\-sprintf}!table.h@{table.h}}
\subsubsection{\setlength{\rightskip}{0pt plus 5cm}int cell\_\-sprintf ({\bf T\_\-cell} $\ast$ {\em c}, {\bf T\_\-zstring} $\ast$ {\em z}, {\bf T\_\-cell\_\-fmt} $\ast$ {\em fmt})}\label{table_8h_a26}


sprintf the cell to the passed zstring using the given fmt structure Note: will alloc zstring as needed. \begin{Desc}
\item[Parameters: ]\par
\begin{description}
\item[{\em 
c}]the cell to sprintf \item[{\em 
z}]the T\_\-zstring:: to hold the result (automatically resized) \item[{\em 
fmt}]the cell format to use, or NULL for to use {\bf T\_\-cell\_\-class::fmt} {\rm (p.\,\pageref{structS__cell__class_m0})} \end{description}
\end{Desc}
\begin{Desc}
\item[Returns: ]\par
the return value sprintf (number of characters formatted) \end{Desc}
\index{table.h@{table.h}!table_create@{table\_\-create}}
\index{table_create@{table\_\-create}!table.h@{table.h}}
\subsubsection{\setlength{\rightskip}{0pt plus 5cm}{\bf T\_\-table}$\ast$ table\_\-create (size\_\-t {\em rows}, size\_\-t {\em cols})}\label{table_8h_a27}


The creator for T\_\-table::\begin{Desc}
\item[Parameters: ]\par
\begin{description}
\item[{\em 
rows}]the number of rows for the table \item[{\em 
cols}]the number of columns for the table \end{description}
\end{Desc}
\begin{Desc}
\item[Returns: ]\par
the T\_\-table:: created \end{Desc}
\index{table.h@{table.h}!table_destroy@{table\_\-destroy}}
\index{table_destroy@{table\_\-destroy}!table.h@{table.h}}
\subsubsection{\setlength{\rightskip}{0pt plus 5cm}void table\_\-destroy ({\bf T\_\-table} $\ast$ {\em t})}\label{table_8h_a28}


The destroyer for T\_\-table::\begin{Desc}
\item[Parameters: ]\par
\begin{description}
\item[{\em 
t}]the table to destroy \end{description}
\end{Desc}
\index{table.h@{table.h}!table_get_cell@{table\_\-get\_\-cell}}
\index{table_get_cell@{table\_\-get\_\-cell}!table.h@{table.h}}
\subsubsection{\setlength{\rightskip}{0pt plus 5cm}{\bf T\_\-cell}$\ast$ table\_\-get\_\-cell ({\bf T\_\-table} $\ast$ {\em t}, size\_\-t {\em row}, size\_\-t {\em col})}\label{table_8h_a29}


Look up the cell based on the row, colum pair in a T\_\-table::\begin{Desc}
\item[Parameters: ]\par
\begin{description}
\item[{\em 
t}]the table \item[{\em 
row}]the row \item[{\em 
col}]the column \end{description}
\end{Desc}
\index{table.h@{table.h}!table_print@{table\_\-print}}
\index{table_print@{table\_\-print}!table.h@{table.h}}
\subsubsection{\setlength{\rightskip}{0pt plus 5cm}void table\_\-print ({\bf T\_\-table} $\ast$ {\em t}, FILE $\ast$ {\em out}, {\bf T\_\-table\_\-fmt} $\ast$ {\em fmt})}\label{table_8h_a36}


print the table using the current format\begin{Desc}
\item[Parameters: ]\par
\begin{description}
\item[{\em 
t}]the table to print \item[{\em 
out}]the file to which to print \item[{\em 
fmt}]the fmt for the table (use g\_\-table\_\-class::fmt if NULL) \end{description}
\end{Desc}
\index{table.h@{table.h}!table_reformat@{table\_\-reformat}}
\index{table_reformat@{table\_\-reformat}!table.h@{table.h}}
\subsubsection{\setlength{\rightskip}{0pt plus 5cm}{\bf T\_\-cell}$\ast$ table\_\-reformat ({\bf T\_\-table} $\ast$ {\em t}, size\_\-t {\em row}, size\_\-t {\em col}, {\bf T\_\-cell\_\-fmt} $\ast$ {\em fmt})}\label{table_8h_a35}


Reformat a cell from it's native type to a string using sprintf and the passed format table. \begin{Desc}
\item[Parameters: ]\par
\begin{description}
\item[{\em 
t}]the table \item[{\em 
row}]the row \item[{\em 
col}]the column \item[{\em 
fmt}]the fmt structure \end{description}
\end{Desc}
\index{table.h@{table.h}!table_set_cell@{table\_\-set\_\-cell}}
\index{table_set_cell@{table\_\-set\_\-cell}!table.h@{table.h}}
\subsubsection{\setlength{\rightskip}{0pt plus 5cm}{\bf T\_\-cell}$\ast$ table\_\-set\_\-cell ({\bf T\_\-table} $\ast$ {\em t}, size\_\-t {\em row}, size\_\-t {\em col}, {\bf T\_\-cell\_\-type} {\em type}, const void $\ast$ {\em data})}\label{table_8h_a30}


Store a value in a cell

This is a convenience function for strings as a commonly used type\begin{Desc}
\item[Parameters: ]\par
\begin{description}
\item[{\em 
t}]the table \item[{\em 
row}]the row \item[{\em 
col}]the column \item[{\em 
type}]the type of data \item[{\em 
data}]the address of the data to store (except for type VOID where the parameter is the data to store). ZSTRING are copied into a T\_\-zstring::.\end{description}
\end{Desc}
\begin{Desc}
\item[Returns: ]\par
the cell set or NULL if row or column out-of-bounds \end{Desc}
\index{table.h@{table.h}!table_set_cell_dbl@{table\_\-set\_\-cell\_\-dbl}}
\index{table_set_cell_dbl@{table\_\-set\_\-cell\_\-dbl}!table.h@{table.h}}
\subsubsection{\setlength{\rightskip}{0pt plus 5cm}{\bf T\_\-cell}$\ast$ table\_\-set\_\-cell\_\-dbl ({\bf T\_\-table} $\ast$ {\em t}, size\_\-t {\em row}, size\_\-t {\em col}, double {\em dbl})}\label{table_8h_a32}


Store a double in cell(row,col) of the T\_\-table::

This is a convenience function for double as a commonly used type\begin{Desc}
\item[Parameters: ]\par
\begin{description}
\item[{\em 
t}]the table \item[{\em 
row}]the row \item[{\em 
col}]the column \item[{\em 
dbl}]the value to store\end{description}
\end{Desc}
\begin{Desc}
\item[Returns: ]\par
the cell set or NULL if row or column out-of-bounds \end{Desc}
\index{table.h@{table.h}!table_set_cell_u32@{table\_\-set\_\-cell\_\-u32}}
\index{table_set_cell_u32@{table\_\-set\_\-cell\_\-u32}!table.h@{table.h}}
\subsubsection{\setlength{\rightskip}{0pt plus 5cm}{\bf T\_\-cell}$\ast$ table\_\-set\_\-cell\_\-u32 ({\bf T\_\-table} $\ast$ {\em t}, size\_\-t {\em row}, size\_\-t {\em col}, unsigned long {\em u32})}\label{table_8h_a34}


Store a unsigned long in cell(row,col) of the T\_\-table::

This is a convenience function for unsigned long long as a commonly used type\begin{Desc}
\item[Parameters: ]\par
\begin{description}
\item[{\em 
t}]the table \item[{\em 
row}]the row \item[{\em 
col}]the column \item[{\em 
u32}]the value to store\end{description}
\end{Desc}
\begin{Desc}
\item[Returns: ]\par
the cell set or NULL if row or column out-of-bounds \end{Desc}
\index{table.h@{table.h}!table_set_cell_u64@{table\_\-set\_\-cell\_\-u64}}
\index{table_set_cell_u64@{table\_\-set\_\-cell\_\-u64}!table.h@{table.h}}
\subsubsection{\setlength{\rightskip}{0pt plus 5cm}{\bf T\_\-cell}$\ast$ table\_\-set\_\-cell\_\-u64 ({\bf T\_\-table} $\ast$ {\em t}, size\_\-t {\em row}, size\_\-t {\em col}, unsigned long long {\em u64})}\label{table_8h_a33}


Store a unsigned long long in cell(row,col) of the T\_\-table::

This is a convenience function for unsigned long long as a commonly used type\begin{Desc}
\item[Parameters: ]\par
\begin{description}
\item[{\em 
t}]the table \item[{\em 
row}]the row \item[{\em 
col}]the column \item[{\em 
u64}]the value to store\end{description}
\end{Desc}
\begin{Desc}
\item[Returns: ]\par
the cell set or NULL if row or column out-of-bounds \end{Desc}
\index{table.h@{table.h}!table_set_cell_zstr@{table\_\-set\_\-cell\_\-zstr}}
\index{table_set_cell_zstr@{table\_\-set\_\-cell\_\-zstr}!table.h@{table.h}}
\subsubsection{\setlength{\rightskip}{0pt plus 5cm}{\bf T\_\-cell}$\ast$ table\_\-set\_\-cell\_\-zstr ({\bf T\_\-table} $\ast$ {\em t}, size\_\-t {\em row}, size\_\-t {\em col}, const char $\ast$ {\em zstr})}\label{table_8h_a31}


Store a NULL terminated string in a cell

This is a convenience function for strings as a commonly used type\begin{Desc}
\item[Parameters: ]\par
\begin{description}
\item[{\em 
t}]the table \item[{\em 
row}]the row \item[{\em 
col}]the column \item[{\em 
zstr}]the NULL terminated string to store (the string is copied)\end{description}
\end{Desc}
\begin{Desc}
\item[Returns: ]\par
the cell set or NULL if row or column out-of-bounds \end{Desc}


\subsection{Variable Documentation}
\index{table.h@{table.h}!g_cell_class@{g\_\-cell\_\-class}}
\index{g_cell_class@{g\_\-cell\_\-class}!table.h@{table.h}}
\subsubsection{\setlength{\rightskip}{0pt plus 5cm}{\bf T\_\-cell\_\-class} g\_\-cell\_\-class ()}\label{table_8h_a6}


The class object for T\_\-cell:: \index{table.h@{table.h}!g_cell_fmt_default@{g\_\-cell\_\-fmt\_\-default}}
\index{g_cell_fmt_default@{g\_\-cell\_\-fmt\_\-default}!table.h@{table.h}}
\subsubsection{\setlength{\rightskip}{0pt plus 5cm}{\bf T\_\-cell\_\-fmt} g\_\-cell\_\-fmt\_\-default ()}\label{table_8h_a5}


default print formats

These formats defined the output for T\_\-cell:: fields. \index{table.h@{table.h}!g_table_class@{g\_\-table\_\-class}}
\index{g_table_class@{g\_\-table\_\-class}!table.h@{table.h}}
\subsubsection{\setlength{\rightskip}{0pt plus 5cm}{\bf T\_\-table\_\-class} g\_\-table\_\-class ()}\label{table_8h_a13}


The class data for T\_\-table:: \index{table.h@{table.h}!g_table_fmt_csv@{g\_\-table\_\-fmt\_\-csv}}
\index{g_table_fmt_csv@{g\_\-table\_\-fmt\_\-csv}!table.h@{table.h}}
\subsubsection{\setlength{\rightskip}{0pt plus 5cm}{\bf T\_\-table\_\-fmt} g\_\-table\_\-fmt\_\-csv ()}\label{table_8h_a12}


The csv formatting for a T\_\-table:: \index{table.h@{table.h}!g_table_fmt_html@{g\_\-table\_\-fmt\_\-html}}
\index{g_table_fmt_html@{g\_\-table\_\-fmt\_\-html}!table.h@{table.h}}
\subsubsection{\setlength{\rightskip}{0pt plus 5cm}{\bf T\_\-table\_\-fmt} g\_\-table\_\-fmt\_\-html ()}\label{table_8h_a11}


Formatting for a T\_\-table:: which will produce an HTML table 