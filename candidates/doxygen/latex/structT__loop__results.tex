\section{T\_\-loop\_\-results Struct Reference}
\label{structT__loop__results}\index{T_loop_results@{T\_\-loop\_\-results}}
stores the table of results for counts and tests. The results are organized into lines ({\bf T\_\-loop\_\-result\_\-line} {\rm (p.\,\pageref{structT__loop__result__line})}::) for each valid count in the range defined by size, start, and incr. Additionally, the run field controls whether a given test is run at all. 


\subsection*{Public Attributes}
\begin{CompactItemize}
\item 
size\_\-t {\bf size}
\item 
size\_\-t {\bf start}
\item 
size\_\-t {\bf incr}
\item 
{\bf T\_\-stats\_\-record} $\ast$ {\bf loop\_\-overhead}
\item 
{\bf T\_\-loop\_\-result\_\-line} $\ast$ {\bf lines}
\item 
char {\bf run} [NUM\_\-TEST]
\end{CompactItemize}


\subsection{Detailed Description}
stores the table of results for counts and tests. The results are organized into lines ({\bf T\_\-loop\_\-result\_\-line} {\rm (p.\,\pageref{structT__loop__result__line})}::) for each valid count in the range defined by size, start, and incr. Additionally, the run field controls whether a given test is run at all.



\subsection{Member Data Documentation}
\index{T_loop_results@{T\_\-loop\_\-results}!incr@{incr}}
\index{incr@{incr}!T_loop_results@{T\_\-loop\_\-results}}
\subsubsection{\setlength{\rightskip}{0pt plus 5cm}size\_\-t T\_\-loop\_\-results::incr}\label{structT__loop__results_m2}


count number increment \index{T_loop_results@{T\_\-loop\_\-results}!lines@{lines}}
\index{lines@{lines}!T_loop_results@{T\_\-loop\_\-results}}
\subsubsection{\setlength{\rightskip}{0pt plus 5cm}{\bf T\_\-loop\_\-result\_\-line}$\ast$ T\_\-loop\_\-results::lines}\label{structT__loop__results_m4}


the results line for each count \index{T_loop_results@{T\_\-loop\_\-results}!loop_overhead@{loop\_\-overhead}}
\index{loop_overhead@{loop\_\-overhead}!T_loop_results@{T\_\-loop\_\-results}}
\subsubsection{\setlength{\rightskip}{0pt plus 5cm}{\bf T\_\-stats\_\-record}$\ast$ T\_\-loop\_\-results::loop\_\-overhead}\label{structT__loop__results_m3}


the data for empty loop time \index{T_loop_results@{T\_\-loop\_\-results}!run@{run}}
\index{run@{run}!T_loop_results@{T\_\-loop\_\-results}}
\subsubsection{\setlength{\rightskip}{0pt plus 5cm}char T\_\-loop\_\-results::run[NUM\_\-TEST]}\label{structT__loop__results_m5}


run this test \index{T_loop_results@{T\_\-loop\_\-results}!size@{size}}
\index{size@{size}!T_loop_results@{T\_\-loop\_\-results}}
\subsubsection{\setlength{\rightskip}{0pt plus 5cm}size\_\-t T\_\-loop\_\-results::size}\label{structT__loop__results_m0}


number of results lines \index{T_loop_results@{T\_\-loop\_\-results}!start@{start}}
\index{start@{start}!T_loop_results@{T\_\-loop\_\-results}}
\subsubsection{\setlength{\rightskip}{0pt plus 5cm}size\_\-t T\_\-loop\_\-results::start}\label{structT__loop__results_m1}


starting count number 

The documentation for this struct was generated from the following file:\begin{CompactItemize}
\item 
/users/jcrouse/perforce/drivers/projects/geodelx/linux/glibc/candidates/common/utils/{\bf loop\_\-test.c}\end{CompactItemize}
